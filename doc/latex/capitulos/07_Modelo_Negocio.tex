\chapter {Modelo de negocio}

\section{Introducción}
        \begin{text}
                Este proyecto intenta abordar los problemas de infraestructura que puede tener una empresa que se dedique al desarrollo de aplicaciones web. Soluciona este problema dando una solución efectiva y asequible económicamente. También soluciona los problemas que pueden tener empresas de desarrollo software, donde no existen unos cauces de desarrollo bien definidos. Esto lo hace un producto ideal tanto para pequeñas startups como para empresas que se han quedado un poco desfasadas en cuanto a los cauces de producción software.
        \end{text}

\section{Inversión inicial}
        \begin{text}
                Vamos a definir el capital, tanto humano como económico necesario para convertir este proyecto en una pequeña empresa que provee servicios a otras empresas tecnológicas.

                \begin{itemize}
                        \item Constitución SL: \textbf{3000} \euro.
                        \item Equipo para desarrollador: \textbf{2000} \euro.
                        \item Equipo para comercial: \textbf{1800} \euro.
                        \item Sueldo comercial: \textbf{18000} \euro.
                        \item Sueldo ingeniero: \textbf{20000} \euro.
                \end{itemize}

                El primer año la creación y mantenimiento del producto supondría un coste para la empresa de \textbf{44.800} \euro.
        \end{text}

\section{Hosting para empresas}
        \begin{text}
        		Aunque se trate de una solución de código abierto, este proyecto se podría vender a empresas de desarrollo web como un servicio de infraestructura. Aquí el producto no sería el proyecto como tal si no la instalación y el mantenimiento de este. Es una buena alternativa a las soluciones cloud de hoy en día como AWS, Azure, Google Cloud... Como hemo visto en la sección \nameref{comparacion_precios} supone un ahorro económico notable.
        \end{text}

\section{Aplicación cauces desarrollo para empresas}
\begin{text}
        Este proyecto no consiste únicamente en despliegue y creación de infraestructura. También aplica los principios DevOps a la creación de nuevo software. Es por esto que otro posible producto de la empresa, podría ser la implementación de las nuevas técnicas DevOps en empresas tecnológicas que hayan quedado desfasadas, con el fin de agilizar y asegurar la creación del software. La idea es hacer una auditoría a las empresas, haciéndoles ver la cantidad de tiempo y esfuerzos que están dedicando al desarrollo del software y como se agilizaría implementando cauces de integración y entrega continua.
\end{text}