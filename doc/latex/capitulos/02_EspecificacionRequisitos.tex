\chapter {Análisis del sistema}

\section{Casuísticas que se dan y como se resuelve cada una}
		\begin{itemize}
			\item \textbf{Problema 1}: \textit{Recreación ante fallo total o réplica de infraestructura}. Ante un fallo total de la infraestructura o replicación para propósitos de testing, con los sistemas actuales sería imposible, ya que dada la configuración manual de cada uno de los servidores, sería imposible replicar al 100\% el estado de los servidores. 
			
			\item \textbf {Solución 1}: Al adoptar IaaC, la infraestructura estará codificada en ficheros de configuración bajo control de versiones. Pudiendo recrear la infraestructura bajo demanda en cualquier momento y asegurándonos de que sea la misma al 100%.
			
			\item \textbf{Problema 2}: \textit{Diferencias entre servidor de desarrollo y producción}. Al desarrollar en local, los desarrolladores no tienen una réplica de los servidores de producción para poder probar sus cambios. Esto hace que haya discrepancias entre Sistemas y desarrolladores puesto que puede funcionar en local con una configuración dada pero no en producción. 
			
			\item \textbf {Solución 2}: Al estar la infraestructura de cada aplicación codificada y bajo control de versiones, será sencillo replicar el entorno de producción en un entorno test para el desarrollo de aplicaciones.
			
			\item \textbf{Problema 3}: \textit{Petición de subidas a producción}. Al no existir cauces de integración o despliegue, no se tiene un conocimiento exacto de qué comportamiento va a tener la aplicación en producción. Esto y la necesidad de hacer peticiones al equipo de sistemas para subir nuevas versiones a producción ralentizan el desarrollo del software.
			
			\item \textbf{Solución 3:} Al tener el desarrollador un servidor réplica de producción para desplegar las aplicaciones, sabe perfectamente el comportamiento que va a tener en producción, puesto que son entornos idénticos. Esto y los cauces CI-CD solucionan el problema.
			
			\item \textbf{Problema 4}: \textit{Desconocimiento estado infraestructura o servicios}. Al configurarse los servidores de forma manual, parche tras parche, es imposible conocer en qué estado se encuentra la infraestructura. También dificulta la replicación de esta para procesos de pruebas.
			
			\item \textbf{Solución 4:} Misma solución que Problema 1. Al adoptar IaaC, la infraestructura estará codificada en ficheros de configuración bajo control de versiones. Esto facilita saber qué paquetes hay instalados en el sistema y qué configuración se ha desplegado para cada servicio.	
		\end{itemize}


\section{Subjetivos, requisitos funcionales, restricciones e issues}
	\begin{text}
		En el capítulo 1, introducción se definen y detallan los objetivos que este proyecto pretende alcanzar. Sin embargo, estos objetivos presentan un alto grado de abstracción y es difícil trasladarlos a tareas concretas que se traducirán en issues del proyecto. En esta sección vamos a traducir los objetivos principales del proyecto en Milestones, desgranar los objetivos en subobjetivos y traducir cada subobjetivo a un requisito funcional que se convertirá en un issue en el proyecto.
	\end{text}

	\subsection{Subobjetivos}
	\label{subobjetivos}
		\begin{text}
			Vamos a analizar cada uno de los objetivos principales con el fin de desgranarlos en subobjetivos que se traduzcan en tareas realizables. 
			\begin{itemize}
				\item \textbf{Objetivo 1}. Asegurar unos tiempos bajos en la creación de infraestructura para proveer servicios.
				\begin{itemize}
					\item \textbf{Subobjetivo 1.1}. Automatización despliegue infraestructura sobre bare metal (tanto este subobjetivo como sus tareas se deben realizar de forma automática con alguna herramienta de automatización).
					\begin{itemize}
						\item Instalar ISO sobre bare metal.
						\item Configuración sistema.
						\item Instalar servicio de virtualización.
						\item Configurar servicio virtualización.
						\item Crear cluster usando el servicio de virtualización.
						\item Agregar los nodos al cluster.
						\item Configurar sistemas de almacenamiento en cluster.
					\end{itemize}
				\end{itemize}
				\item \textbf{Objetivo 2}. Replicar la infraestructura asegurando bajos tiempos de creación. El objetivo 1 cubre las necesidades del objetivo 2. Al crear la infraestructura con herramientas de automatización y estando la infraestructura codificada en ficheros de configuración bajo control de versiones, aseguramos poder replicar la infraestructura en cualquier momento. Al automatizarlo aseguramos mejores tiempos que de forma manual.
				
				\item \textbf{Objetivo 3}. Crear servicios necesarios para una empresa de aplicaciones web asegurando bajos tiempos de creación.
				\begin{itemize}
					\item \textbf{Subobjetivo 3.1}. Automatización despliegue servicios en infraestructura (tanto este subobjetivo como sus tareas se deben realizar de forma automática con alguna herramienta de automatización).
					\begin{itemize}
						\item Desplegar firewall redundante.
						\item Desplegar servicio de control de versiones.
						\item Desplegar servicio de almacenamiento estático.
						\item Desplegar servicio gestión aplicaciones.
						\item Desplegar servicio orquestación contenedores.
						\item Desplegar servicio de servidor web.
						\item Desplegar servicio de base de datos.
						\item Desplegar servicio de monitorización.
						\item Desplegar servicio gestor de colas.
					\end{itemize}
				\end{itemize}
				\item \textbf{Objetivo 4}. Replicar cualquier servicio alojado en la infraestructura de una empresa en la infraestructura de forma rápida. Al igual que pasa con los objetivos 1 y 2, los objetivos 3 y 4 están relacionados y si se cumple 3 se cumple 4. Si conseguimos automatizar la creación de los servicios con ficheros de configuración bajo control de versiones, aseguramos el poder replicarlos en cualquier momento.
				\item \textbf{Objetivo 5}. Crear cauces rápidos y seguros para la creación de nuevo software.
				\begin{itemize}
					\item \textbf{Subobjetivo 5.1}. Automatización despliegue aplicaciones con integración continua.
					\begin{itemize}
						\item Instalación y configuración servicio CI + CD.
						\item Adaptación proyecto a CI/CD (tests).
						\item Creación de los cauces en servicio elegido.
					\end{itemize}
				\end{itemize}
			\end{itemize}
		\end{text}
	
	\subsection{Requisitos funcionales}
	\label{requisitosfuncionales}
	\begin{itemize}
		\item \textbf{RF 0.} El sistema permitirá la automatizar la configuración del equipo del sistema que va a utilizar el proyecto para la configuración de sistemas remotos. \label{RF0}
		\item \textbf{RF 1.} El sistema permitirá la automatizar la creación de la infraestructura. \label{RF1}
		\begin{itemize}
			\item \textbf{RF 1.1}. El sistema permitirá automatizar ĺa instalación de una ISO elegida sobre el servidor.
			\item \textbf{RF 1.2}. El sistema permitirá automatizar ĺa instalación de una ISO elegida sobre el servidor.
			\item \textbf{RF 1.3}. El sistema permitirá la instalación de un servicio de virtualización. 
			\item \textbf{RF 1.4}. El sistema permitirá la configuración de un servicio de virtualización.
			\item \textbf{RF 1.5}. El sistema permitirá la creación de un cluster sobre el servicio de virtualización.
			\item \textbf{RF 1.6}. El sistema permitirá añadir nuevos nodos al cluster.
			\item \textbf{RF 1.7}. El sistema permitirá configurar los sistemas de almacenamiento utilizados por el cluster.
		\end{itemize}
		\item \textbf{RF 2.} El sistema permitirá la replicación de la infraestructura. \label{RF2}
		\item \textbf{RF 3.} El sistema permitirá la creación de los servicios necesarios para una empresa de desarrollo web. \label{RF3}
		\begin{itemize}
			\item \textbf{RF 3.1}. El sistema permitirá la creación de un servicio de firewall en el cluster. (RNF 4)
			\item \textbf{RF 3.2}. El sistema permitirá la creación de un servicio de control de versiones en el cluster.
			\item \textbf{RF 3.3}. El sistema permitirá la creación de un servicio de almacenamiento estático en el cluster.
			\item \textbf{RF 3.4}. El sistema permitirá la creación de un servicio de gestión de aplicaciones Wordpress en el cluster.
			\item \textbf{RF 3.5}. El sistema permitirá la creación de un servicio de orquestación de contenedores en el cluster. (RNF 4)
			\item \textbf{RF 3.6}. El sistema permitirá la creación de un servicio de servidor web en el cluster.
			\item \textbf{RF 3.7}. El sistema permitirá la creación de un servicio de monitorización en el cluster.
			\item \textbf{RF 3.8}. El sistema permitirá la creación de un servicio de base de datos en el cluster. (RNF 3)
			\item \textbf{RF 3.9}. El sistema permitirá la creación de un servicio de gestión de colas.
			\item \textbf{RF 3.10}. El sistema permitirá la creación de un servicio de orquestación de contenedores.
		\end{itemize}
	
	\item \textbf{RF 4}. El sistema permitirá la recreación de infraestructura en un tiempo bajo. \label{RF4}
	
	\item \textbf{RF 5}. El sistema permitirá el despliegue automatizados en un cluster seguro. \label{RF5}
	\end{itemize}

	\subsection{Requisitos no funcionales}
	\begin{itemize}
		\item \textbf{RNF 1}. El sistema debe garantizar la automatización de los procesos de creación de infraestructura asegurando unos tiempos bajos.
		\item \textbf{RNF 2}. El sistema debe garantizar la automatización de los procesos de creación de servicios en la infraestructura asegurando unos tiempos bajos.
		\item \textbf{RNF 3}. El sistema debe garantizar la alta disponibilidad de el servicio de base de datos.
		\item \textbf{RNF 4}. El sistema debe garantizar la alta disponibilidad de las aplicaciones.
	\end{itemize}

	\subsection{Restricciones}
	\label{restricciones}
	\begin{text}
		Este proyecto se ha desarrollado para una empresa en concreto y se realiza sobre una base. Esto conlleva restricciones a la hora de elegir las tecnologías a utilizar para satisfacer los objetivos. A continuación se listan las restricciones tecnológicas que presenta este proyecto. Cabe destacar que la mayoría de tecnologías a utilizar son impuestas ya que la infraestructura ya existe y el objetivo de este proyecto es la automatización de la misma.
		
		\begin{itemize}
			\item \textbf{Restricción 1.} El uso de las siguientes tecnologías será necesario para la correcta adaptación del proyecto a la empresa:
			\begin{itemize}
				\item \textbf{RS 1.1}. Bases de datos: MySQL
				\item \textbf{RS 1.2}. Almacenamiento estático: Ceph cluster
				\item \textbf{RS 1.3}. Servicio virtualización: Proxmox
				\item \textbf{RS 1.4}. Servido web: Nginx
				\item \textbf{RS 1.5}. Contenedores: Docker
				\item \textbf{RS 1.6}. Firewall: pfSense
				\item \textbf{RS 1.7}. Control de versiones: GitLab
				\item \textbf{RS 1.8}. Proyectos: Django + Angular \& Wordpress
				\item \textbf{RS 1.9}. Contenedores: Docker
			\end{itemize}
		\end{itemize}
	\end{text}

\section{Casos de uso}
\label{casosdeuso}
	\subsection{CU 0. Configurar entorno proyecto}
	\href{https://github.com/VictorMorenoJimenez/tfg2020/milestone/11?closed=1}{Issues en Github}.
	
	\begin{text}
		Caso de uso 0. Define los pasos a seguir para configurar el entorno donde se va a ejecutar el proyecto. Las condiciones que se deben cumplir para que se de el caso de uso y el estado final tras ejecutar el caso de uso.
	\end{text}

	\begin{usecase}{Configurar entorno proyecto << CU.0 >>}
		\addrow{Descripción}{Configura el equipo donde se van a lanzar las tareas del proyecto.}
		
		\addrow{Actores}{Administrador/Desarrollador}
		
		\addrow{Precondición}{El usuario dispone de un equipo con una ISO instalada de la familia Debian.}
		
		\addrow{Postcondición}{El equipo del usuario queda preparado para la ejecución de tareas del proyecto.}
		
		\addmulrow{Secuencia principal (P)}{
			\item El usuario clona el proyecto
			\item El usuario configura las va- \\
			riables necesarias para ejecutar el \\ 
			playbook
			\item El usuario ejecuta el playbook \\ \textbf{configure\_development\_\\environment} 
			\item El sistema ejecuta las tareas en el \\
			equipo local
			\item El sistema informa al usuario \\
			del resultado de la ejecución}
		\hline
	\end{usecase}

	\hyperref[RF0]{Requisito funcional 0}
	\clearpage
	
	\subsection{CU 1. Crear infraestructura}
	\href{https://github.com/VictorMorenoJimenez/tfg2020/milestone/6}{Issues en Github}.
	
	\begin{text}
		Caso de uso 1. Define los pasos a seguir para crear la infraestructura, las condiciones que se deben cumplir para que se de el caso de uso y el estado final tras ejecutar el caso de uso.
	\end{text}

	\begin{usecase}{Crear infraestructura << CU.1 >>}
		\addrow{Descripción}{Permite crear la infraestructura sobre los nodos.}
		
		\addrow{Actores}{Administrador}
		
		\addrow{Precondición}{El usuario administrador ha clonado el proyecto y ha configurado correctamente los ficheros hosts.yml, asignado valor a las variables de los roles hetzner y proxmox\_nodes y ejecutado caso de uso 0}
		
		\addrow{Postcondición}{Los nodos están configurados con la ISO elegida instalada y el servicio de virtualización Proxmox desplegado. Creado y configurado cluster Proxmox.}
		
		\addmulrow{Secuencia principal (P)}{
			\item El administrador clona el proyecto
			\item El administrador configura las va- \\
				riables necesarias para ejecutar el \\ 
				playbook
			\item El administrador ejecuta el play- \\
			book \textbf{configure\_proxmox\_nodes} 
			\item El sistema ejecuta las tareas en los \\
				nodos 
			\item El sistema informa al administrador \\
				  del resultado de la ejecución}
			  \hline
	\end{usecase}

	\hyperref[RF1]{Requisito funcional 1}
	\clearpage
	\subsection{CU.2. Replicar infraestructura}
	\href{https://github.com/VictorMorenoJimenez/tfg2020/milestone/6}{Issues en Github}.

	\begin{text}
		Caso de uso 2. Define los pasos a seguir para replicar la infraestructura, las condiciones que se deben cumplir para que se de el caso de uso y el estado final tras ejecutar el caso de uso.
	\end{text}

	\begin{usecase}{Replicar infraestructura << CU.2 >>}
		\addrow{Descripción}{Permite replicar la infraestructura sobre los nodos.}
		
		\addrow{Actores}{Administrador}
		
		\addrow{Precondición}{El usuario administrador ha clonado el proyecto y ha configurado correctamente el fichero hosts.yml con los nuevos nodos donde replicar la infraestructura, asignado valor a las variables de los roles hetzner y proxmox\_nodes y ejecutado caso de uso 0}
		
		\addrow{Postcondición}{La infraestructura desplegada en << CU.1 >> queda replicada en los nodos elegidos.}
		
		\addmulrow{Secuencia principal (P)}{
			\item El administrador clona el proyecto
			\item El administrador configura las va- \\
			riables necesarias para ejecutar el \\ 
			playbook
			\item El administrador ejecuta el play- \\
			book \textbf{configure\_proxmox\_nodes} 
			\item El sistema ejecuta las tareas en los \\
			nodos 
			\item El sistema informa al administrador \\
			del resultado de la ejecución}
 
		\hline
	\end{usecase}

	\hyperref[RF2]{Requisito funcional 2}

	\clearpage
	
	\subsection{CU.3. Crear servicio firewall}
	\href{https://github.com/VictorMorenoJimenez/tfg2020/milestone/7}{Issues en Github}
	
	\begin{text}
		Caso de uso 3. Define los pasos a seguir para crear un servicio de firewall en la infraestructura, las condiciones que se deben cumplir para que se de el caso de uso y el estado final tras ejecutar el caso de uso.
	\end{text}

	\begin{usecase}{Crear servicio firewall << CU.3 >>}
		\addrow{Descripción}{Permite crear un servicio de firewall en el cluster.}
		
		\addrow{Actores}{Administrador}
		
		\addrow{Precondición}{El usuario ha ejecutado correctamente << CU.0 >> y  << CU.1 >> o << CU.2 >>}
		
		\addrow{Postcondición}{Se ha creado el servicio de firewall pfSense en el cluster.}
		
		\addmulrow{Secuencia principal (P)}{
			\item El administrador ejecuta \\ correctamente << CU.0 >>
			\item El administrador ejecuta \\ correctamente  << CU.1 >> \\
			o << CU.2 >>.
			\item El administrador configura \\ variables 
			de role \textbf{pfsense} 
			\item El administrador ejecuta playbook \\  
				\textbf{deploy\_pfsense\_firewall}
			\item El sistema ejecuta las tareas en los \\
			nodos. 
			\item El sistema informa al administrador \\
			del resultado de la ejecución}
		\hline
	\end{usecase}

\hyperref[RF3]{Requisito funcional 3}

\clearpage
\subsection{CU.4. Crear servicio almacenamiento estático}
\href{https://github.com/VictorMorenoJimenez/tfg2020/issues/133}{Github Issue}

\begin{text}
	Caso de uso 4. Define los pasos a seguir para crear un servicio de almacenamiento estático, las condiciones que se deben cumplir para que se de el caso de uso y el estado final tras ejecutar el caso de uso.
\end{text}
\begin{usecase}{Crear servicio almacenamiento estático << CU.4 >>}
	\addrow{Descripción}{Permite crear un servicio de almacenamiento estático en el cluster.}
	
	\addrow{Actores}{Administrador}
	
	\addrow{Precondición}{El usuario ha ejecutado correctamente << CU.0 >> y  << CU.1 >> o << CU.2 >> y << CU.3 >>}
	
	\addrow{Postcondición}{Se ha creado el servicio de almacenamiento estático \textbf{Ceph cluster} en el cluster.}
	
	\addmulrow{Secuencia principal (P)}{
		\item El administrador ejecuta \\ correctamente << CU.0 >>
		\item El administrador ejecuta \\ correctamente  << CU.1 >> \\
		o << CU.2 >> y << CU.3 >>.
		\item El administrador configura \\ variables 
		de role \textbf{pfsense} 
		\item El administrador ejecuta playbook \\  
		\textbf{deploy\_pfsense\_firewall}
		\item El sistema ejecuta las tareas en los \\
		nodos. 
		\item El sistema informa al administrador \\
		del resultado de la ejecución}
	\hline
\end{usecase}

\hyperref[RF3]{Requisito funcional 3}

\clearpage

\subsection{CU.5. Crear servicio de gestión de aplicaciones Wordpress}

\href{https://github.com/VictorMorenoJimenez/tfg2020/issues/135}{Github Issue}

\begin{text}
	Caso de uso 5. Define los pasos a seguir para crear un servicio de gestión de aplicaciones Wordpress, las condiciones que se deben cumplir para que se de el caso de uso y el estado final tras ejecutar el caso de uso.
\end{text}

\begin{usecase}{Crear servicio de gestión de aplicaciones Wordpress << CU.5 >>}
	\addrow{Descripción}{Permite crear un servicio de gestión de aplicaciones Wordpress en el cluster.}
	
	\addrow{Actores}{Administrador}
	
	\addrow{Precondición}{El usuario ha ejecutado correctamente << CU.0 >> y  << CU.1 >> o << CU.2 >> y << CU.3 >>}
	
	\addrow{Postcondición}{Se ha creado el servicio de gestión de aplicaciones Wordpress en el cluster.}
	
	\addmulrow{Secuencia principal (P)}{
		\item El administrador ejecuta \\ correctamente << CU.0 >>
		\item El administrador ejecuta \\ correctamente  << CU.1 >> \\
		o << CU.2 >> y << CU.3 >>.
		\item El administrador configura \\ variables 
		de role \textbf{virtualmin} 
		\item El administrador ejecuta playbook \\  
		\textbf{create\_virtualmin}
		\item El sistema ejecuta las tareas en los \\
		nodos. 
		\item El sistema informa al administrador \\
		del resultado de la ejecución}
	\hline
\end{usecase}

\hyperref[RF3]{Requisito funcional 3}

\clearpage 
\subsection{CU.6. Crear servicio de orquestación de contenedores en el cluster}
\href{https://github.com/VictorMorenoJimenez/tfg2020/issues/136}{Github Issue}

\begin{text}
	Caso de uso 6. Define los pasos a seguir para crear un servicio orquetación de contenedores, las condiciones que se deben cumplir para que se de el caso de uso y el estado final tras ejecutar el caso de uso.
\end{text}

\begin{usecase}{Crear servicio de orquestación de contenedores en el cluster << CU.6 >>}
	\addrow{Descripción}{Permite crear un servicio de orquestación de contenedores en el cluster en el cluster.}
	
	\addrow{Actores}{Administrador}
	
	\addrow{Precondición}{El usuario ha ejecutado correctamente << CU.0 >> y  << CU.1 >> o << CU.2 >> y << CU.3 >>}
	
	\addrow{Postcondición}{Se ha creado el servicio de orquestación de contenedores en el cluster.}
	
	\addmulrow{Secuencia principal (P)}{
		\item El administrador ejecuta \\ correctamente << CU.0 >>
		\item El administrador ejecuta \\ correctamente  << CU.1 >> \\
		o << CU.2 >> y << CU.3 >>.
		\item El administrador configura \\ variables 
		de role \textbf{kubernetes} 
		\item El administrador ejecuta playbook \\  
		\textbf{create\_kubernetes\_cluster}
		\item El sistema ejecuta las tareas en los \\
		nodos. 
		\item El sistema informa al administrador \\
		del resultado de la ejecución}
	\hline
\end{usecase}

\hyperref[RF3]{Requisito funcional 3}


\clearpage
\subsection{CU.7. Crear servicio de monitorización en el cluster}
\href{https://github.com/VictorMorenoJimenez/tfg2020/issues/58}{Github Issue}

\begin{text}
	Caso de uso 7. Define los pasos a seguir para crear un servicio de monitorización del cluster, las condiciones que se deben cumplir para que se de el caso de uso y el estado final tras ejecutar el caso de uso.
\end{text}

\begin{usecase}{Crear servicio de monitorización en el cluster << CU.7 >>}
	\addrow{Descripción}{Permite crear un servicio de monitorización en el cluster.}
	
	\addrow{Actores}{Administrador}
	
	\addrow{Precondición}{El usuario ha ejecutado correctamente << CU.0 >> y  << CU.1 >> o << CU.2 >> y << CU.3 >>}
	
	\addrow{Postcondición}{Se ha creado el servicio de monitorización en el cluster.}
	
	\addmulrow{Secuencia principal (P)}{
		\item El administrador ejecuta \\ correctamente << CU.0 >>
		\item El administrador ejecuta \\ correctamente  << CU.1 >> \\
		o << CU.2 >> y << CU.3 >>.
		\item El administrador configura \\ variables 
		de role \textbf{icinga2}.
		\item El administrador ejecuta playbook \\  
		\textbf{create\_icinga2}.
		\item El sistema ejecuta las tareas en los \\
		nodos. 
		\item El sistema informa al administrador \\
		del resultado de la ejecución}
	\hline
\end{usecase}

\hyperref[RF3]{Requisito funcional 3}

\clearpage
\subsection{CU.8. Crear servicio de bases de datos}
\href{https://github.com/VictorMorenoJimenez/tfg2020/issues/134}{Github Issue}

\begin{text}
	Caso de uso 8. Define los pasos a seguir para crear un servicio de base de datos en el cluster, las condiciones que se deben cumplir para que se de el caso de uso y el estado final tras ejecutar el caso de uso.
\end{text}

\begin{usecase}{Crear servicio de bases de datos << CU.8 >>}
	\addrow{Descripción}{Permite crear un servicio bases de datos en el cluster.}
	
	\addrow{Actores}{Administrador}
	
	\addrow{Precondición}{El usuario ha ejecutado correctamente << CU.0 >> y  << CU.1 >> o << CU.2 >> y << CU.3 >>}
	
	\addrow{Postcondición}{Se ha creado el servicio de bases de datos en el cluster.}
	
	\addmulrow{Secuencia principal (P)}{
		\item El administrador ejecuta \\ correctamente << CU.0 >>
		\item El administrador ejecuta \\ correctamente  << CU.1 >> \\
		o << CU.2 >> y << CU.3 >>.
		\item El administrador configura \\ variables 
		de role \textbf{icinga2} 
		\item El administrador ejecuta playbook \\  
		\textbf{create\_galera\_cluster}
		\item El sistema ejecuta las tareas en los \\
		nodos. 
		\item El sistema informa al administrador \\
		del resultado de la ejecución}
	\hline
\end{usecase}

\hyperref[RF3]{Requisito funcional 3}


\clearpage 
\subsection{CU.10. Crear servicio de gestión de colas}
\href{https://github.com/VictorMorenoJimenez/tfg2020/issues/137}{Github Issue}

\begin{text}
	Caso de uso 10. Define los pasos a seguir para crear un servicio de gestión de colas en el cluster, las condiciones que se deben cumplir para que se de el caso de uso y el estado final tras ejecutar el caso de uso.
\end{text}

\begin{usecase}{Crear servicio de gestión de colas << CU.10 >>}
	\addrow{Descripción}{Permite crear un servicio gestión de colas en el cluster.}
	
	\addrow{Actores}{Administrador}
	
	\addrow{Precondición}{El usuario ha ejecutado correctamente << CU.0 >> y  << CU.1 >> o << CU.2 >> y << CU.3 >>}
	
	\addrow{Postcondición}{Se ha creado el servicio de gestión de colas en el cluster.}
	
	\addmulrow{Secuencia principal (P)}{
		\item El administrador ejecuta \\ correctamente << CU.0 >>
		\item El administrador ejecuta \\ correctamente  << CU.1 >> \\
		o << CU.2 >> y << CU.3 >>.
		\item El administrador configura \\ variables 
		de role \textbf{rabbitmq} 
		\item El administrador ejecuta playbook \\  
		\textbf{create\_rabbitmq}
		\item El sistema ejecuta las tareas en los \\
		nodos. 
		\item El sistema informa al administrador \\
		del resultado de la ejecución}
	\hline
\end{usecase}

\hyperref[RF3]{Requisito funcional 3}

\clearpage

\subsection{CU.11. Desplegar aplicación en cluster}

\href{https://github.com/VictorMorenoJimenez/tfg2020/milestone/12}{Github Issues}.

\begin{text}
	Caso de uso 11. Define los pasos a seguir para desplegar una aplicación en el cluster, las condiciones que se deben cumplir para que se de el caso de uso y el estado final tras ejecutar el caso de uso.
\end{text}

\begin{usecase}{Desplegar aplicación << CU.12 >>}
	\addrow{Descripción}{Permite desplegar la aplicación en el cluster.}
	
	\addrow{Actores}{Administrador}
	
	\addrow{Precondición}{El usuario ha ejecutado correctamente << CU.0 >> y  << CU.1 >> o << CU.2 >> y << CU.3 >> y el cluster está desplegado. Aplicación Django + Angular dockerizada}.
	
	\addrow{Postcondición}{La aplicación queda desplegada en el cluster.}
	
	\addmulrow{Secuencia principal (P)}{
		\item El administrador ejecuta \\ correctamente << CU.0 >>
		\item El administrador ejecuta \\ correctamente  << CU.1 >> \\
		o << CU.2 >> y << CU.3 >>.
		\item El usuario sube cambios al \\ sistema de control de versiones. 
		\item El sistema ejecuta el cauce para \\ compilar la aplicación.   
		\item El sistema ejecuta el cauce \\ para desplegar la nueva \\ versión en el docker swarm.}
	\hline
\end{usecase}

\hyperref[RF5]{Requisito funcional 5}

\clearpage

	
	
	
