\chapter {Modelo de negocio}

\section{Introducción}
	\begin{text}
		Este proyecto intenta abordar los problemas de infraestructura que puede tener una empresa que se dedique al desarrollo de aplicaciones web. Soluciona este problema dando una solución efectiva y asequible económicamente. También soluciona los problemas que pueden tener empresas de desarrollo software, donde no existen unos cauces de desarrollo bien definidos. Esto lo hace un producto ideal tanto para pequeñas startups como para empresas que se han quedado un poco desfasadas en cuanto a los cauces de producción software.
	\end{text}

\section{Inversión inicial}
	\begin{text}
		Vamos a definir el capital, tanto humano como económico necesario para convertir este proyecto en una pequeña empresa que provee servicios a otras empresas tecnológicas.
		
		\begin{itemize}
			\item Constitución SL: \textbf{3000} \euro.
			\item Equipo para desarrollador: \textbf{2000} \euro.
			\item Equipo para comercial: \textbf{1800} \euro.
			\item Sueldo comercial: \textbf{18000} \euro.
			\item Sueldo ingeniero: \textbf{20000} \euro.
		\end{itemize}
	
		El primer año la creación y mantenimiento del producto supondría un coste para la empresa de \textbf{44.800} \euro.
	\end{text}

\section{Hosting para empresas}
	\begin{text}
		Explicamos cómo podemos proveer infraestructura para empresas que se dediquen a desarrollo de software, alternativas a soluciones cloud.
		Enfocado a empresas de desarrollo web. Se podría vender el proyecto solucionando el problema de creación y mantenimiento de la infraestructura de una empresa basada en desarrollo web o desarrollo software en general.
	\end{text}

\section{Aplicación cauces desarrollo para empresas}
\begin{text}
	Este proyecto no consiste únicamente en despliegue y creación de infraestructura. También aplica los principios DevOps a la creación de nuevo software. Es por esto que otro posible producto de la empresa, podría ser la implementación de las nuevas técnicas DevOps en empresas tecnológicas que hayan quedado desfasadas, con el fin de agilizar y asegurar la creación del software.
\end{text}