\chapter {Conclusiones}


\section{Futuro del proyecto}
\begin{text}
	A continuación se hace un pequeño análisis de cómo puede evolucionar el proyecto en un futuro. De tener éxito y intentar aplicarlo a empresas, hay muchos aspectos del proyecto que se pueden mejorar o añadir. 
\end{text}
\subsection{Adaptación proyecto distintas plataformas}
\begin{text}
	Una de las principales desventajas que tiene este proyecto, es que está limitado su instalación y uso al sistema operativo Debian 10 buster. Se podría ampliar a otros entornos basados en debian pero actualmente, únicamente está comprobado y probado que funciona en Debian 10 buster. \\
	En caso de que el proyecto prospere, se debería ampliar la compatibilidad con distintas distribuciones de Linux como Ubuntu o CentOS que son de las más usadas. De esta forma se podría llegar a más público y hacer más atractivo el proyecto.
\end{text}
\subsection{Creación WUI para el despliegue de playbooks}
\begin{text}
	Dado el carácter técnico del proyecto, un usuario normal que utilizara este proyecto como servicio, tendría que aprender como funcionan los playbooks de Ansible en los que están basados este proyecto... Esto no es muy atractivo para posibles usuarios, ya que tendrían que tener conocimiento de cómo funcionan los playbooks, los roles... Para solventar el problema anterior, se sugiere la implementación de una interfaz de usuario, mediante la cual poder elegir que tareas ejecutar. Por ejemplo, se podrían crear formularios para insertar las variables necesarias y un selector donde elegir la tarea a desplegar. De esta forma el usuario no tendría que conocer como funcionan los playbooks o los roles de Ansible, simplemente mediante una interfaz intuitiva sabría que hace cada uno pero no como lo hacen. \\
	Esta parte del futuro del proyecto me parece especialmente interesante ya que una interfaz de usuario hace muy atractivo el producto. A continuación se muestra una posible interfaz
\end{text}
