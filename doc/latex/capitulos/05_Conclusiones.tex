\chapter {Conclusiones}

\section{Grado de cumplimiento de los objetivos propuestos}
\begin{itemize}
	\item \textbf{Objetivo 1}. Asegurar unos tiempos bajos en la creación de infraestructura para proveer servicios. Gracias a los playbooks de Ansible hemos conseguido crear la infraestructura base donde desplegar los servicios necesarios con un solo playbook. En la subsección \nameref{analisis_tiempos} hacemos una comparativa de los tiempos antes y después de este proyecto para confirmar que, efectivamente se ha logrado el objetivo.
	\item \textbf{Objetivo 1.1}. Infraestructura segura frente ataques. Gracias a los firewall pfSense, conseguimos un cluster seguro, donde bloqueamos todo el tráfico menos el tráfico dentro de la red LAN. 
	\item \textbf{Objetivo 2}. Replicar infraestructura asegurando bajos tiempos de creación. Como hemos visto, el Objetivo 1 se ha conseguido utilizando un playbook de Ansible. Esto hace que inmediatamente se cumpla el objetivo 2, ya que este playbook nos va a permitir replicar la infraestructura en nuevos nodos. De nuevo en la sección \nameref{analisis_tiempos} podemos confirmar que hemos mejorado sustancialmente los tiempos de creación de infraestructura.
	\item \textbf{Objetivo 3}. Crear y desplegar servicios necesarios para una empresa de aplicaciones web asegurando bajos tiempos de creación. Todos los servicios necesarios para este proyecto, han sido desplegados y configurados con playbooks de Ansible. Esto permite crear y recrear los servicios en cualquier momento, asegurando un estado definido de cada servicio. El análisis de los tiempos se efectúa en la sección \nameref{analisis_tiempos}.
	\item \textbf{Objetivo 4}. Crear cauces rápidos y seguros para la creación de nuevo software. Con la ayuda de GitLab CI/CD y Docker, hemos conseguido crear un cauce de integración continua y despliegue continuo donde desplegar de forma segura los nuevos cambios en las aplicaciones. También se efectúa un análisis de la mejora en tiempos respecto a antes del proyecto en la sección \nameref{analisis_tiempos}.
\end{itemize}

\section{Análisis mejora tiempos}
\label{analisis_tiempos} 
\subsection{Tiempo estimado creación infraestructura antes del proyecto}

\subsection{Tiempo estimado creación infraestructura después del proyecto}
\subsection{Tiempo estimado creación servicios antes del proyecto}
\subsection{Tiempo estimado creación servicios después del proyecto}
\subsection{Tiempo estimado integración nuevos cambios en aplicación antes del proyecto}
\subsection{Tiempo estimado integración nuevos cambios en aplicación después del proyecto}


\section{Futuro del proyecto}
\begin{text}
	A continuación se hace un pequeño análisis de cómo puede evolucionar el proyecto en un futuro. De tener éxito y intentar aplicarlo a empresas, hay muchos aspectos del proyecto que se pueden mejorar o añadir. 
\end{text}
\subsection{Adaptación proyecto distintas plataformas}
\begin{text}
	Una de las principales desventajas que tiene este proyecto, es que está limitado su instalación y uso al sistema operativo Debian 10 buster. Se podría ampliar a otros entornos basados en debian pero actualmente, únicamente está comprobado y probado que funciona en Debian 10 buster. \\
	En caso de que el proyecto prospere, se debería ampliar la compatibilidad con distintas distribuciones de Linux como Ubuntu o CentOS que son de las más usadas. De esta forma se podría llegar a más público y hacer más atractivo el proyecto.
\end{text}
\subsection{Creación WUI para el despliegue de playbooks}
\begin{text}
	Dado el carácter técnico del proyecto, un usuario normal que utilizara este proyecto como servicio, tendría que aprender como funcionan los playbooks de Ansible en los que están basados este proyecto... Esto no es muy atractivo para posibles usuarios, ya que tendrían que tener conocimiento de cómo funcionan los playbooks, los roles... Para solventar el problema anterior, se sugiere la implementación de una interfaz de usuario, mediante la cual poder elegir que tareas ejecutar. Por ejemplo, se podrían crear formularios para insertar las variables necesarias y un selector donde elegir la tarea a desplegar. De esta forma el usuario no tendría que conocer como funcionan los playbooks o los roles de Ansible, simplemente mediante una interfaz intuitiva sabría que hace cada uno pero no como lo hacen. \\
	Esta parte del futuro del proyecto me parece especialmente interesante ya que una interfaz de usuario hace muy atractivo el producto. A continuación se muestra una posible interfaz
\end{text}
