\chapter {Anexo I: Manual Usuario}
	\section{Despliegue automático sobre bare metal}
		\subsection{Requisitos previos}
			\subsubsection{Hetzner}
				\begin{paragraph}
					Explicamos que es Hetzner y mencionamos capítulos anteriores.
						\begin{itemize}
							\item Ip adicional nodos maestros.
							\item MAC adicional para ip adicional nodos maestros.
							\item IP failover.
							\item Virtual Switch conexión nodos en cluster. 
							\item Clave pública. Fingerprint
						\end{itemize}	
				\end{paragraph}
			\subsubsection{Dominio}
				\begin{paragraph}
					Explicamos la necesidad de tener comprado un dominio para todos los servicios. La necesidad de crear los siguientes CNAMES dentro de la zona DNS del dominio comprado:
					\begin{itemize}
						\item nodo1.dominio.com
						\item nodo2.dominio.com 
						\item nodo3.dominio.com
						\item gitab.dominio.com
						\item pfsense-01.dominio.com
						\item pfsense-02.dominio.com
						\item failover.dominio.com
						\item ceph-admin.dominio.com
						\item rabbitmq-01.dominio.com
					\end{itemize}
				\end{paragraph}
			\subsubsection{Ansible Hosts}
			\subsubsection{Dependencias. Configuración host}
				\begin{paragraph}
					Lista de dependencias que hay que instalar en el host donde se lanzan los playbooks.
					Comprobar todos los modulos de Ansible que utilizamos para ver sus dependencias y agregarlas aqui.
					\begin{itemize}
						\item Ansible  2.9.11
						\item Python  3.7
						\item Proxmoxer pip
						\item requests pip 
						\item dependencias pip varias.
						\item Molecule
						\item Docker
					\end{itemize}
				\end{paragraph}
			\subsubsection{Clonar repositorio desde Github}
			
		\subsection{Configuración variables Playbook}
		\subsection{Ejecución Playbook}
	\section{Despliegue automático y aprovisionamiento de máquinas virtuales}
		\subsection{pfSense}
		\subsection{GitLab}
			\subsubsection{GitLab Runner}
		\subsection{Webproxy}
		\subsection{Docker Swarm}
		\subsection{Kubernetes}
		\subsection{Galera cluster}
		\subsection{Ceph}
		\subsection{RabbitMQ}
		\subsection{Virtualmin}
		