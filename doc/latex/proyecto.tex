\documentclass[a4paper,11pt]{book}
%\documentclass[a4paper,twoside,11pt,titlepage]{book}
\usepackage{listings}
\usepackage[utf8]{inputenc}
\usepackage[spanish]{babel}
\usepackage{amsfonts}
\usepackage{booktabs}
\usepackage{siunitx}
\usepackage[official]{eurosym}
% \usepackage[style=list, number=none]{glossary} %
%\usepackage{titlesec}
%\usepackage{pailatino}

\usepackage{listings}
\usepackage{color}

\definecolor{dkgreen}{rgb}{0,0.6,0}
\definecolor{gray}{rgb}{0.5,0.5,0.5}
\definecolor{mauve}{rgb}{0.58,0,0.82}

\lstset{frame=tb,
	language=Java,
	aboveskip=3mm,
	belowskip=3mm,
	showstringspaces=false,
	columns=flexible,
	basicstyle={\small\ttfamily},
	numbers=none,
	numberstyle=\tiny\color{gray},
	keywordstyle=\color{blue},
	commentstyle=\color{dkgreen},
	stringstyle=\color{mauve},
	breaklines=true,
	breakatwhitespace=true,
	tabsize=3
}

\decimalpoint
\usepackage{dcolumn}
\newcolumntype{.}{D{.}{\esperiod}{-1}}
\makeatletter
\addto\shorthandsspanish{\let\esperiod\es@period@code}
\makeatother


%\usepackage[chapter]{algorithm}
\RequirePackage{verbatim}
%\RequirePackage[Glenn]{fncychap}
\usepackage{fancyhdr}
\usepackage{graphicx}
\usepackage{afterpage}

\usepackage{longtable}

\usepackage[pdfborder={000}]{hyperref} %referencia

% ********************************************************************
% Re-usable information
% ********************************************************************
\newcommand{\myTitle}{TFG\xspace}
\newcommand{\myDegree}{Grado en Ingeniería Informática\xspace}
\newcommand{\myName}{Víctor Moreno Jiménez \xspace}
\newcommand{\myProf}{Juan Julián Melero Guervós \xspace}
%\newcommand{\mySupervisor}{Put name here\xspace}
\newcommand{\myFaculty}{Escuela Técnica Superior de Ingenierías Informática y de
Telecomunicación\xspace}
\newcommand{\myFacultyShort}{E.T.S. de Ingenierías Informática y de
Telecomunicación\xspace}
\newcommand{\myDepartment}{Departamento de Sistemas\xspace}
\newcommand{\myUni}{\protect{Universidad de Granada}\xspace}
\newcommand{\myLocation}{Granada\xspace}
\newcommand{\myTime}{\today\xspace}
\newcommand{\myVersion}{Versión 0.1\xspace}


\hypersetup{
pdfauthor = {\myName victormoreno@correo.ugr.es},
pdftitle = {\myTitle},
pdfsubject = {},
pdfkeywords = {devOps, integración continuo, despliegue continúo},
pdfcreator = {LaTeX con el paquete pdflatex},
pdfproducer = {pdflatex}
}

%\hyphenation{}


%\usepackage{doxygen/doxygen}
%\usepackage{pdfpages}
\usepackage{url}
\usepackage{colortbl,longtable}
\usepackage[stable]{footmisc}
%\usepackage{index}

%\makeindex
%\usepackage[style=long, cols=2,border=plain,toc=true,number=none]{glossary}
% \makeglossary

% Definición de comandos que me son tiles:
%\renewcommand{\indexname}{Índice alfabético}
%\renewcommand{\glossaryname}{Glosario}

\pagestyle{fancy}
\fancyhf{}
\fancyhead[LO]{\leftmark}
\fancyhead[RE]{\rightmark}
\fancyhead[RO,LE]{\textbf{\thepage}}
\renewcommand{\chaptermark}[1]{\markboth{\textbf{#1}}{}}
\renewcommand{\sectionmark}[1]{\markright{\textbf{\thesection. #1}}}

\setlength{\headheight}{1.5\headheight}

\newcommand{\HRule}{\rule{\linewidth}{0.5mm}}
%Definimos los tipos teorema, ejemplo y definición podremos usar estos tipos
%simplemente poniendo \begin{teorema} \end{teorema} ...
\newtheorem{teorema}{Teorema}[chapter]
\newtheorem{ejemplo}{Ejemplo}[chapter]
\newtheorem{definicion}{Definición}[chapter]

\definecolor{gray97}{gray}{.97}
\definecolor{gray75}{gray}{.75}
\definecolor{gray45}{gray}{.45}
\definecolor{gray30}{gray}{.94}

\lstset{ frame=Ltb,
     framerule=0.5pt,
     aboveskip=0.5cm,
     framextopmargin=3pt,
     framexbottommargin=3pt,
     framexleftmargin=0.1cm,
     framesep=0pt,
     rulesep=.4pt,
     backgroundcolor=\color{gray97},
     rulesepcolor=\color{black},
     %
     stringstyle=\ttfamily,
     showstringspaces = false,
     basicstyle=\scriptsize\ttfamily,
     commentstyle=\color{gray45},
     keywordstyle=\bfseries,
     %
     numbers=left,
     numbersep=6pt,
     numberstyle=\tiny,
     numberfirstline = false,
     breaklines=true,
   }
 
% minimizar fragmentado de listados
\lstnewenvironment{listing}[1][]
   {\lstset{#1}\pagebreak[0]}{\pagebreak[0]}

\lstdefinestyle{CodigoC}
   {
	basicstyle=\scriptsize,
	frame=single,
	language=C,
	numbers=left
   }
\lstdefinestyle{CodigoC++}
   {
	basicstyle=\small,
	frame=single,
	backgroundcolor=\color{gray30},
	language=C++,
	numbers=left
   }

 
\lstdefinestyle{Consola}
   {basicstyle=\scriptsize\bf\ttfamily,
    backgroundcolor=\color{gray30},
    frame=single,
    numbers=none
   }


\newcommand{\bigrule}{\titlerule[0.5mm]}

\usepackage{booktabs}

\newcommand\tabularhead[1]{
	\begin{table}[!hbt]
		\caption{Acción #1}
		
		\begin{tabular}{|p{0.4\linewidth}|p{0.55\linewidth}|}
			\hline
			\textbf{Acción} & \textbf{#1} \\
			\hline}
		
		\newcommand\addrow[2]{#1 &#2\\ \hline}
		
		\newcommand\addmulrow[2]{ \begin{minipage}[t][][t]{2.5cm}#1\end{minipage}% 
			&\begin{minipage}[t][][t]{8cm}
				\begin{enumerate} #2   \end{enumerate}
			\end{minipage}\\ }
		
		\newenvironment{usecase}{\tabularhead}
{\hline\end{tabular}\end{table}}



%Para conseguir que en las páginas en blanco no ponga cabecerass
\makeatletter
\def\clearpage{%
  \ifvmode
    \ifnum \@dbltopnum =\m@ne
      \ifdim \pagetotal <\topskip
        \hbox{}
      \fi
    \fi
  \fi
  \newpage
  \thispagestyle{empty}
  \write\m@ne{}
  \vbox{}
  \penalty -\@Mi
}
\makeatother

\usepackage{pdfpages}
\begin{document}
\begin{titlepage}
 
 
\newlength{\centeroffset}
\setlength{\centeroffset}{-0.5\oddsidemargin}
\addtolength{\centeroffset}{0.5\evensidemargin}
\thispagestyle{empty}

\noindent\hspace*{\centeroffset}\begin{minipage}{\textwidth}

\centering
\includegraphics[width=0.9\textwidth]{imagenes/logo_ugr.jpg}\\[1.4cm]

\textsc{ \Large TRABAJO FIN DE GRADO\\[0.2cm]}
\textsc{ INGENIERÍA EN INFORMÁTICA}\\[1cm]
% Upper part of the page
% 
% Title
{\Huge\bfseries Despliegue automático de infraestructura + CI/CD \\
}
\noindent\rule[-1ex]{\textwidth}{3pt}\\[3.5ex]
{\large\bfseries Implementación de técnicas DevOps}
\end{minipage}

\vspace{2.5cm}
\noindent\hspace*{\centeroffset}\begin{minipage}{\textwidth}
\centering

\textbf{Autor}\\ {Víctor Moreno Jiménez (alumno)}\\[2.5ex]
\textbf{Directores}\\
{Juan Julián Merelo Guervós}\\[2.5ex]
\includegraphics[width=0.3\textwidth]{imagenes/etsiit_logo.png}\\[0.1cm]
\textsc{Escuela Técnica Superior de Ingenierías Informática y de Telecomunicación}\\
\textsc{---}\\
Granada, Junio de 2020
\end{minipage}
%\addtolength{\textwidth}{\centeroffset}
%\vspace{\stretch{2}}
\end{titlepage}



\chapter*{}
%\thispagestyle{empty}
%\cleardoublepage

%\thispagestyle{empty}

\input{portada/portada_2}


\cleardoublepage
\thispagestyle{empty}

\begin{center}
{\large\bfseries  Despliegue automático de infraestructura + CI/CD: Implementación de técnicas DevOps}\\
\end{center}
\begin{center}
Víctor Moreno Jiménez\\
\end{center}

%\vspace{0.7cm}
\noindent{\textbf{Palabras clave}: DevOps, IaC (Infraestructure as Code), CI (Continous Integration), CD (Continous Delivery)}\\

\vspace{0.7cm}
\noindent{\textbf{Resumen}}\\

El objectivo de este TFG es conseguir un ciclo rápido y replicable de la infraestructura de una pequeña empresa así como conseguir crear un ciclo de creación y despliegue de aplicaciones. 

\bigskip 
En concreto se va a automatizar todo el proceso de despliegue de infraestructura (IaC) y se van a aplicar los principios de integración contínua y entrega contínua propios de la filosofía DevOps en todo el ciclo de desarrollo de los productos de la empresa. La finalidad de esto es poder disponer de una infraestructura segura y facilmente recreable en caso de catástrofe o ampliación del cluster. Desde el punto de vista de los desarrolladores de la empresa, podrán tener a su disposición y bajo demanda, un entorno de desarrollo mediante ficheros de configuración sin depender directamente del departamento de sistemas. \bigskip

Este proyecto supone una mejora técnica en la creación y despliegue de infraestructura de la empresa.

\cleardoublepage


\thispagestyle{empty}


\begin{center}
{\large\bfseries Automatic deployment of infrastructure + CI / CD: Implementation of DevOps techniques}\\
\end{center}
\begin{center}
Víctor Moreno Jiménez\\
\end{center}

%\vspace{0.7cm}
\noindent{\textbf{Keywords}: DevOps, IaC (Infraestructure as Code), CI (Continous Integration), CD (Continous Delivery)}\\

\vspace{0.7cm}
\noindent{\textbf{Abstract}}\\

The main goal of this project is to achieve a fast and replicable cycle of infraestructure in a small company, as well as to create a cycle of creation and deployment of applications. \bigskip

The entire company infraestructure will be automated using some orchestration tools via configuration files (Infrastructure as Code). Continuous integration and continuous deployment will be used to create development pipelines. \bigskip

The purpose of this is to be able to have a secure and easily deployable infrastructure. From the point of view of the company's developers, they will have the ability to create development infrastructures on demand without depending directly on the systems department.

This project is a technical improvement in creation and deployment of the infraestructure of the company.
\bigskip


\chapter*{}
\thispagestyle{empty}

\noindent\rule[-1ex]{\textwidth}{2pt}\\[4.5ex]

Yo, \textbf{Victor Moreno Jiménez}, alumno de la titulación Ingeniería Informática de la \textbf{Escuela Técnica Superior
de Ingenierías Informática y de Telecomunicación de la Universidad de Granada}, con DNI 47252942V, autorizo la
ubicación de la siguiente copia de mi Trabajo Fin de Grado en la biblioteca del centro para que pueda ser consultada por las personas que lo deseen.

\vspace{6cm}

\noindent Fdo: Víctor Moreno Jiménez

\vspace{2cm}

\begin{flushright}
Granada a 31 de Julio de 2020.
\end{flushright}


\chapter*{}
\thispagestyle{empty}

\noindent\rule[-1ex]{\textwidth}{2pt}\\[4.5ex]

D. \textbf{Juan Julián Merelo Guervós}, Profesor del Área de XXXX del Departamento YYYY de la Universidad de Granada.


\vspace{0.5cm}

\textbf{Informan:}

\vspace{0.5cm}

Que el presente trabajo, titulado \textit{\textbf{Automatic deployment of infrastructure + CI / CD: Implementation of DevOps}},
ha sido realizado bajo su supervisión por \textbf{Víctor Moreno Jiménez}, y autorizamos la defensa de dicho trabajo ante el tribunal
que corresponda.

\vspace{0.5cm}

Y para que conste, expiden y firman el presente informe en Granada a 25 de Agosto de 2020.

\vspace{1cm}

\textbf{Los directores:}

\vspace{5cm}
\noindent Fdo: Juan Julián Merelo Guervós



\chapter*{Agradecimientos}
\thispagestyle{empty}

       \vspace{1cm}


Para toda la gente maravillosa que me he cruzado en estos 4 años de carrera. No habría sido lo mismo sin vosotros... Gracias


\frontmatter
\tableofcontents
\listoffigures
\listoftables
%
\mainmatter
\setlength{\parskip}{5pt}

\chapter {Introducción}
\label{capitulo1}
	\begin{paragraph}
		Actualmente trabajo como administrador de sistemas en una empresa de desarrollo de aplicaciones web. Este proyecto es una colaboración con la empresa con el fin de mejorar técnicamente toda la infraestructura necesaria para desarrollar la actividad de la empresa. Principalmente una mejora en los tiempos de creación/replicación de infraestructura y en los cauces de desarrollo de software. A continuación se detalla el alcance del proyecto así como una definición específica del problema que se pretende resolver.
	\end{paragraph}

\section{Accesibilidad del proyecto}
\begin{text}
	Este proyecto ha sido desarrollado bajo la licencia \textbf{GNU General Public License v3.0} y es de carácter público. Se puede acceder a este a través de GitHub en el siguiente \href{https://github.com/VictorMorenoJimenez/tfg2020}{enlace}. Cualquiera puede contribuir al código a través de un pull request. También forma parte de los \href{https://github.com/JJ/TF-libres-UGR}{trabajos liberados} de la UGR.
\end{text}
\section{Definición del problema}
	\begin{text}
		Actualmente la empresa para la que trabajo dispone de una infraestructura desplegada en un proveedor de servidores bare metal, esta infraestructura se ha ido creando con el tiempo, añadiendo servicios y haciendo las modificaciones pertinentes de forma manual. \\
		Dicha infraestructura es la responsable de alojar todos los servicios que ofrecen a los clientes y se utiliza también para el desarrollo de nuevo software. Los desarrolladores utilizan máquinas virtuales para simular entornos de producción donde desplegar las aplicaciones en fase de pruebas antes de hacer un despliegue definitivo en producción. \\
		Esto plantea varios problemas:
		\begin{itemize}
			\item Infraestructura configurada manualmente, imposible de replicar rápidamente.
			\item Configuración de los distintos servicios manual, imposible de replicar rápidamente. 
			\item Difícil conocer el estado actual de los distintos servicios desplegados. 
			\item Imposible replicar la infraestructura en tiempos asequibles ante fallo total.
			\item Los desarrolladores dependen del equipo de sistemas para incorporar cambios en aplicaciones en pruebas.
			\item Ralentización de los procesos de desarrollo de software debido a la dependencia del equipo de sistemas.
		\end{itemize}
	\end{text}

\section{Objetivos}
\label{objetivos_primarios}
\begin{text}
	Una vez identificados los problemas podemos definir una serie de objetivos a alcanzar para solucionarlos. Este proyecto pretende lograr:
	\begin{itemize}
		\item \textbf{Objetivo 1}. Asegurar unos tiempos bajos en la creación de infraestructura para proveer servicios.
		\item \textbf{Objetivo 2}. Replicar la infraestructura asegurando bajos tiempos de creación.
		\item \textbf{Objetivo 3}. Crear servicios necesarios para una empresa de aplicaciones web asegurando bajos tiempos de creación.
		\item \textbf{Objetivo 4}. Replicar cualquier servicio alojado en la infraestructura de forma rápida.
		\item \textbf{Objetivo 5}. Crear cauces rápidos y seguros para la creación de nuevo software.
	\end{itemize}
\end{text}

\clearpage 

\section{Conceptos básicos}
		\begin{text}
			A continuación se van a describir uno a uno, los conceptos básicos para entender este proyecto. Son conceptos claves sin los cuales un lector no experto en el tema a tratar, no comprenderá ni el problema ni la solución adoptada para el problema.
		\end{text}
	\subsection{DevOps}
		\begin{text}
			El término DevOps es una fusión de las dos palabras Desarrollo y Operaciones. DevOps es una filosofía, una forma de abordar el desarrollo de software. El objetivo de DevOps es fusionar los departamentos desarrollo y operaciones de forma que sea más fácil y rápida la creación de software. La siguiente imagen define el término:
			
			\begin{figure}[!hbt]
				\centering
				\includegraphics[scale=0.75]{imagenes/Introduccion/Conceptos_Basicos/devops.jpg}
				\caption[¿Qué es DevOps?]{¿Qué es DevOps? \cite{WhatIsDe1:online}}
				\label{termino_devops}
			\end{figure}
		\end{text}
	
	\clearpage 
	
	\subsection{Integración Continua (CI)}
		\begin{text}
			La Integración Continua o CI para abreviar, es uno de los pilares de la filosofía DevOps. La integración continua se basa en hacer integraciones automáticas de un proyecto lo más a menudo posible para poder detectar fallos rápidamente. Consta de dos partes: compilación y ejecución de test de un proyecto.
			
			
			\begin{figure}[!hbt]
				\centering
				\includegraphics[scale=0.45]{imagenes/Introduccion/Conceptos_Basicos/ci.png}
				\caption[Integración Continua]{Integración Continua \cite{Alcanzan90:online} }
				\label{integracion_continua} 
			\end{figure}
		\end{text}
	\subsection{Despliegue continuo (CD)}
		\begin{text}
			El despliegue continuo o CD por sus siglas en inglés 'continous delivery' complementa a la integración continua desplegando el proyecto software en los servidores, una vez ha pasado el proceso de la integración continua. Gracias a 'continous delivery' podemos garantizar entregas rápidas y seguras de software.
			
			\begin{figure}[!hbt]
				\centering
				\includegraphics[scale=0.35]{imagenes/Introduccion/Conceptos_Basicos/CD.png}
				\caption[Despliegue Continuo]{Despliegue Continuo \cite{continuo84:online}}
				\label{despligue_continuo} 
			\end{figure}
		\end{text}
	\subsection{Infraestructura como Código (IaaC)}
		\begin{text}
			La infraestructura como código pretende tratar los servidores y toda la infraestructura alrededor de una organización como un software de programación. De este modo, la infraestructura está escrita en ficheros de configuración y es fácilmente replicable y testeable. Este concepto al igual que los dos anteriores está íntimamente ligado con el término DevOps, ya que es uno de los primeros pasos a adoptar. IaaC pretende difuminar la línea entre el código que ejecutan las aplicaciones y el código que configura la infraestructura. Acerca a los desarrolladores al equipo de operaciones o administradores de sistemas. \\
			De esta manera no únicamente se testea el software antes de ser lanzado, si no también la infraestructura. A continuación se muestra como sería el cauce de trabajo siguiendo los principios de IaaC.
			
			\begin{figure}[!hbt]
				\centering
				\includegraphics[scale=0.75]{imagenes/Introduccion/Conceptos_Basicos/IaaC.png}
				\caption[Infraestructura como Código]{Infraestructura como Código \cite{WhatIsIaaC:online}}
				\label{infraestructura_como_codigo} 
			\end{figure}
		\end{text}
	
\section{Solución propuesta}
\begin{text}
	Este proyecto se ha creado para cumplir con los objetivos descritos en la sección ``\nameref{objetivos_primarios}''. A continuación se define a grandes rasgos la solución propuesta que se irá justificando y ampliando a lo largo del documento. \\
	\begin{itemize}
		\item \textbf{Objetivo 1: Asegurar unos tiempos bajos en la creación de infraestructura para proveer servicios}. \\ Aplicando la filosofía DevOps a la infraestructura. Se tratará la infraestructura como si fuese un proyecto software, es decir, la infraestructura será codificada bajo ficheros de configuración y estos estarán alojados en algún sistema de control de versiones.
		\item \textbf{Objetivo 1.1: Infraestructura segura frente a ataques}. \\ Para garantizar la seguridad de la infraestructura, se desplegarán firewalls redundantes. Estos firewalls serán tratados como servicios y entran dentro de las especificaciones del objetivo 3.
		\item \textbf{Objetivo 2: Replicar la infraestructura asegurando bajos tiempos de creación.} \\
		Si se consigue objetivo 1, es decir, tener la infraestructura codificada en ficheros de configuración con alguna herramienta de automatización, esta será fácilmente replicable. Si se cumple el objetivo 1, se cumplirá el objetivo 2.
		\item \textbf{Objetivo 3: Crear y desplegar servicios necesarios para una empresa de aplicaciones web asegurando bajos tiempos de creación.} \\
		Una vez cumplidos objetivos 1 y 2, y con una infraestructura base desplegada, será necesario desplegar los servicios que una empresa, dedicada al desarrollo de aplicaciones web requiere. Servicios como servidores web, servicios de orquestación de contenedores, gestor de paso de mensajes... Para garantizar que estos servicios se pueden crear y desplegar en tiempos eficientes, será necesario utilizar herramientas de automatización, de manera que, mediante ficheros de configuración lanzados en la infraestructura, logramos el despliegue de todos los servicios necesarios.
		\item \textbf{Objetivo 4: Crear cauces rápidos y seguros para la creación de nuevo software.} \\
		Para cumplir con este objetivo se va a instalar un servicio en la infraestructura que permita la creación de dichos cauces. Se aplicarán técnicas para garantizar los despliegues automáticos y la creación de entornos de pruebas para probar las aplicaciones. La seguridad está garantizada ya que todo sucede en un entorno seguro gracias al subobjetivo 1.1, y en una red LAN a la cual únicamente tienen acceso los administradores.
	\end{itemize}
\end{text}


%
\chapter {Análisis del sistema}

\section{Definición del problema}
\begin{paragraph}
	Hoy en día, aún hay muchas compañías que siguen teniendo dos equipos diferenciados dentro el departamento de IT. Sistemas o Operaciones y Desarrolladores. Ya hemos visto en la introducción que esto a la larga causa problemas y no es la forma más rápida y eficiente de desarrollar software. \\
	Trabajo como Administrador de Sistemas en una empresa pequeña y continuamente aparecen problemas. Es por eso que se ha decidido adoptar un enfoque DevOps par el desarrollo del software. En este cambio están implicados ambos departamentos ya que es responsabilidad de ambos que los cauces de desarrollo y despliegue funcionen correctamente. \\
	En mi compañía la Infraestructura donde se alojan los proyectos, se configura manualmente haciendo prácticamente imposible una replicación de ésta en caso de que sea necesario. También las subidas de proyectos y ejecución de tests no están automatizados y cada desarrollador se encarga de ejecutar los tests en local. Ésto implica discrepancias varias entre el equipo de operaciones y el de desarrolladores. Es por esto que se hace necesario la adopción de técnicas DevOps como IaaC, CI y CD en la empresa.
\end{paragraph}
\section{Que se pretende resolver}
		\begin{itemize}
			\item Todos los problemas que involucra la configuración manual de infraestructura.
			\item Ralentización de desarrollo de software, al no tener los cauces automatizados.
		\end{itemize}
\section{Casuísticas que se dan y como se resuelve cada una}
		\begin{itemize}
			\item \textbf{Problema 1}: \textit{Recreación ante fallo total o réplica de infraestructura}. Ante un fallo total de la infraestructura o replicación para propósitos de testing, con los sistemas actuales sería imposible, ya que dada la configuración manual de cada uno de los servidores, sería imposible replicar al 100% el estado de los servidores. 
			\item  \textbf {Solución 1}: Al adoptar IaaC, la infraestructura estará codificada en ficheros de configuración bajo control de versiones. Pudiendo recrear la infraestructura bajo demanda en cualquier momento y asegurándonos de que sea la misma al 100%.
			\item \textbf{Problema 2}: \textit{Diferencias entre servidor de desarrollo y producción}. Al desarrollar en local los desarrolladores, no tienen una réplica de los servidores de producción para poder probar sus cambios. Ésto hace que haya discrepancias entre Sistemas y desarrolladores puesto que puede funcionar en local con una configuración dada pero no en producción. 
			\item \textbf {Solución 2}: Al estar la infraestructura de cada aplicación codificada y bajo control de versiones, será sencillo replicar el entorno de producción en un entorno test para el desarrollo de aplicaciones.
			\item \textbf{Problema 3}: \textit{Petición de subidas a producción}. Al no existir cauces de integración o despliegue, no se tiene un conocimiento exacto de qué comportamiento va a tener la aplicación en producción. Ésto y la necesidad de hacer peticiones al equipo de sistemas para subir nuevas versiones a producción ralentizan el desarrollo del software.
			\item \textbf{Solución 3:} Al tener el desarrollador un servidor réplica de producción para desplegar las aplicaciones, sabe perfectamente el comportamiento que va a tener en producción, puesto que son entornos idénticos. Ésto y los cauces CI-CD solucionan el problema.
		\end{itemize}

\section{Tipos de usuario}
	\begin{paragraph}
		Adoptando la filosofía DevOps, tanto desarrolladores como sysadmin debería adoptar el mismo rol. Sin embargo, en una empresa en la cual crean y configuran sus propios servidores sin contratar servicios externos (como Azure DevOps, AWS...) se crea la necesidad de crear y configurar la infraestructura que va a alojar toda la infraestructura necesaria para el desarrollo cada aplicación. Es por esto que se distinguen entre dos tipos de usuarios: \textbf{Desarrolladores} y \textbf{Sistemas + Desarrolladores}.  
		
		\begin{itemize}
			\item \textbf{Desarrolladores}. Este tipo usuario tiene las siguientes necesidades:
				\item Poder construir una infraestructura para el desarrollo de aplicaciones bajo demanda. [RF 1]
				\item Poder integrar cambios a las aplicaciones. [RF 2]
				\item Poder tener feedback de el estado de la aplicación tras el despliegue. [RF 3]
				\item Conocer la configuración del servidor donde se aloja la aplicación web. [RF 4]
			\item \textbf{Desarrolladores + Sistemas}
				\item Poder construir una infraestructura para el desarrollo de aplicaciones bajo demanda. [RF 1]
				\item Poder integrar cambios a las aplicaciones. [RF 2]
				\item Poder tener feedback de el estado de la aplicación tras el despliegue. [RF 3]
				\item Conocer la configuración del servidor donde se aloja la aplicación web. [RF 4]
				\item Poder desplegar la infraestructura completa que aloja la infraestructura para las aplicaciones. [RF 5]
				\item Poder conocer la configuración de la infraestructura. [RF 6]
		\end{itemize}
	
		Estos historias de usuario representan la funcionalidad final del proyecto, con un nivel de abstracción muy alto. En la siguiente sección, se desgranarán estas historias para definir de forma precisa los requisitos funcionales. \cite{ReqF:online} 
	\end{paragraph}

\section{Especificación de requisitos}
	\label{erf}
	\begin{paragraph}
		A continuación se detallan los requisitos funcionales de este proyecto. 
		\begin{itemize}
			\item Construir y desplegar infraestructura para el desarrollo de aplicaciones bajo demanda. [RF 1]
			\item Poder integrar cambios en las aplicaciones de forma autónoma [RF 2]
			\item Conocer el estado de la aplicación una vez hecho el despliegue [RF 3]
			\item Conocer la configuración del servidor donde se aloja la aplicación web. [RF 4]
			\item Poder desplegar la infraestructura completa que aloja la infraestructura para las aplicaciones a través de ficheros de configuración. [RF 5]
			\item Conocer en todo momento la configuración del servidor [RF 6]
			\item Poder desplegar la infraestructura rápidamente ante un error catastrófico [RF 7]
			\item Realizar tareas de mantenimiento en los servidores con tareas automatizadas [RF 8]
			\item Crear máquinas virtuales bajo demanda con tareas automatizadas y ficheros de configuración [RF 9]
			\item Modificar la configuración de los firewall con tareas automatizadas [RF 10]
		\end{itemize}
	\end{paragraph}

\section{Diagramas}
	\subsection{Diagramas casos de uso}
		\begin{paragraph}
			A continuación se muestran los principales diagramas según los requisitos funcionales descritos en la sección \nameref{erf}.
		\end{paragraph}
	
		\begin{figure}[!hbt]
			\centering
			\includegraphics[scale=0.4]{imagenes/Analisis/casos_uso_administrador.png}
			\caption[Casos de uso Administrador]{Casos de uso \cite{casosuso:online}} 
			\label{Casos de uso Administrador}
		\end{figure}
	
		\begin{figure}[!hbt]
			\centering
			\includegraphics[scale=0.4]{imagenes/Analisis/casos_uso_desarrollador.png}
			\caption[Casos de uso Desarrollador]{Casos de uso \cite{casosuso:online}} 
			\label{Casos de uso Desarrollador}
		\end{figure}
	
	\subsection{Diagramas de secuencia}
		\begin{figure}[!hbt]
			\centering
			\includegraphics[scale=0.4]{imagenes/Analisis/diagrama_secuencia_desarrollador_1.png}
			\caption[Diagrama secuencia Desarrollador 1]{Diagrama secuencia Desarrollador 1 \cite{diagramasecuencia:online}} 
			\label{Diagrama secuencia_desarrollador_1}
		\end{figure}
		\clearpage
	
		\begin{figure}[!hbt]
			\centering
			\includegraphics[scale=0.4]{imagenes/Analisis/diagrama_secuencia_desarrollador_4.png}
			\caption[Diagrama secuencia Desarrollador 2]{Diagrama secuencia Desarrollador 2 \cite{diagramasecuencia:online}} 
			\label{Diagrama secuencia_desarrollador_2}
		\end{figure}

		\begin{figure}[!hbt]
			\centering
			\includegraphics[scale=0.4]{imagenes/Analisis/diagrama_secuencia_desarrollador_3.png}
			\caption[Diagrama secuencia Desarrollador 3]{Diagrama secuencia Desarrollador 3 \cite{diagramasecuencia:online}} 
			\label{Diagrama secuencia_desarrollador_3}
		\end{figure}
	
		\begin{figure}[!hbt]
			\centering
			\includegraphics[scale=0.4]{imagenes/Analisis/diagrama_secuencia_administrador_1.png}
			\caption[Diagrama secuencia Administrador 1]{Diagrama secuencia Administrador 1 \cite{diagramasecuencia:online}}
			\label{Diagrama secuencia_administrador_1}
		\end{figure}
		\clearpage
		
		\begin{figure}[!hbt]
			\centering
			\includegraphics[scale=0.4]{imagenes/Analisis/diagrama_secuencia_administrador_2.png}
			\caption[Diagrama secuencia Administrador 2]{Diagrama secuencia Administrador 2 \cite{diagramasecuencia:online}}
			\label{Diagrama secuencia_administrador_2}
		\end{figure}
	
		\begin{figure}[!hbt]
			\centering
			\includegraphics[scale=0.4]{imagenes/Analisis/diagrama_secuencia_administrador_2.png}
			\caption[Diagrama secuencia Administrador 3]{Diagrama secuencia Administrador 3 \cite{diagramasecuencia:online}}
			\label{Diagrama secuencia_administrador_3}
		\end{figure}
		
	
\section{Arquitectura del Sistema}
	\subsection{Servidores Físicos}
		\begin{paragraph}
			Al trabajar en una empresa de hosting, he tenido la suerte de contar con 3 servidores bare metal para el desarrollo de este proyecto.Las características técnicas de los servidores se pueden consultar en \nameref{servidores_bare_metal}.
		\end{paragraph}
	\subsection{Infraestructura objetivo}
		\label{InfraestructuraObjetivo}
		\begin{paragraph}
			Este proyecto pretende crear una infraestructura robusta para una pequeña empresa que se dedique al desarrollo del software. Ésta infraestructura debe ser robusta al igual que segura, con lo que ha de proporcionar firewalls redundantes y algún mecanismo para proporcional alta disponibilidad en las aplicaciones web.  A continuación se muestra la infraestructura objetivo.
		\end{paragraph}
	
		\begin{figure}[!hbt]
			\centering
			\includegraphics[scale=0.75]{imagenes/Analisis/diagrama.jpg}
			\caption[Infraestructura Objetivo]{Infraestructura Objetivo }
			\label{Infraestructura_objetivo}
		\end{figure}
	\clearpage
	
\section{Metodología de desarrollo}
	\begin{paragraph}
		Todo proyecto software debe tener una organización y unas etapas de desarrollo bien definidas. En esta sección se pretende explicar la metodología de desarrollo elegida para realizar este proyecto. \\
		Se ha tratado este proyecto como cualquier otro proyecto software. Para la organización y el control de versiones se ha elegido Github, un software basado en git originalmente creado para el control de versiones. Actualmente, GitHub ofrece múltiples servicios, como almacenamiento, gestión de paneles de trabajo, registry, integración con múltiples tecnologías... \\
		En cuanto a la metodología de desarrollo, se ha optado por un desarrollo basado en Milestones. Cada Milestone está compuesto por Issues y estos están etiquetados y asignados a personas. A continuación se explican con mayor detalle estos conceptos.
	\end{paragraph}

	\subsection{Milestones}
	\label{milestones}
	\begin{paragraph}
		Los Milestones o Hitos en castellano, corresponden con estados finales deseados de la aplicación. Sabiendo esto, podríamos crear un Milestone por ejemplo: "Servidores configurados a través de ficheros de configuración Ansible". Esto será un estado final deseado para nuestra aplicación o proyecto. Para que un hito quede totalmente realizado, deben estar completos todos los issues marcados como esenciales para el hito. Un hito está compuesto por issues. A continuación se muestran algunos milestones creados en este proyecto, en el panel de administración GitHub.
		
		\begin{figure}[!hbt]
			\centering
			\includegraphics[scale=0.37]{imagenes/Analisis/milestones.png}
			\caption[GitHub milestones]{GitHub milestones}
			\label{github_milestones}
		\end{figure}
	\end{paragraph}
	\subsection{Issues}
	\begin{paragraph}
		Como ya hemos visto, los issues forman parte de los hitos. Es una forma de desgranar el problema. Siguiendo el ejemplo anterior, si tenemos un hito: "Servidores configurados a través de ficheros de configuración Ansible", podemos desgranar el siguiente en distintos issues, que serían tareas más sencillas que hay que realizar para completar el hito. Por ejemplo, algunos issues serían: 
		\begin{itemize}
			\item Crear estructura directorios Ansible.
			\item Instalar paquetes en servidor a través de ficheros de configuración Ansible. \textbf{Core}
			\item Configurar interfaces de red a través de ficheros de configuración Ansible. \textbf{Core}
			\item Instalar ISO en servidor a través de ficheros de configuración Ansible. \textbf{Core}
			\item Instalar certificados SSL a través de ficheros de configuración Ansible. \textbf{Mejora}
		\end{itemize}
	
		Y así seguiríamos creando issues según creamos que van a ser necesarios para completar el hito en cuestión. \\
		En la sección  \nameref{milestones} hemos hablado que los issues tienen etiquetas. En la lista anterior por ejemplo, únicamente tenemos dos etiquetas que nos indican en este caso si son issues imprescindibles para el hito o simplemente mejoras. Gracias a estas etiquetas, podemos distinguir entre distintos tipos de issues y asignar mayor o menos prioridad por etiqueta. También gracias a Github, cada issue puede ser asignado a un desarrollador del proyecto. A continuación se muestra un ejemplo del panel de Github para mostrar los issues, etiquetas y hitos. \\
		Como se puede comprobar, el sistema de etiquetas y asingación de issues a desarrolladores, es más que suficiente para manejar proyectos. Permite asignar prioridades, agrupar issues en hitos y escribir comentarios en cada issue / milestone. 
		
			\begin{figure}[!hbt]
				\centering
				\includegraphics[scale=0.44]{imagenes/Analisis/githubissues.jpg}
				\caption[GitHub panel]{GitHub panel}
				\label{github_issues}
			\end{figure}
	\end{paragraph}
	\clearpage


	
	
	
%
\chapter {Planificación}

\section{Estimación recursos necesarios}
\begin{paragraph}
	En esta sección se va a crear una estimación de los recursos, tanto humanos como económicos que se van a necesitar para llevar a cabo el proyecto. Cabe destacar que en la estimación temporal se incluye un lapso de tiempo para adaptarse a la tecnología a usar. \\ 
	
\end{paragraph}
\section{Estimación temporal}
%
\chapter {Diseño}

\section{Solución adoptada}
	\subsection{Hetzner}
	\subsection{Proxmox}
	\subsection{pfSense}
	\subsection{Kubernetes}
	\subsection{Ceph}
	\subsection{Docker}
	\subsection{Nginx}
	\subsection{GitLab}
\section{Posibles mejoras}
%
\chapter {Conclusiones}

\section{Grado de cumplimiento de los objetivos propuestos}
\begin{itemize}
	\item \textbf{Objetivo 1}. Asegurar unos tiempos bajos en la creación de infraestructura para proveer servicios. Gracias a los playbooks de Ansible hemos conseguido crear la infraestructura base donde desplegar los servicios necesarios con un solo playbook. En la subsección \nameref{analisis_tiempos} hacemos una comparativa de los tiempos antes y después de este proyecto para confirmar que, efectivamente se ha logrado el objetivo.
	\item \textbf{Objetivo 1.1}. Infraestructura segura frente ataques. Gracias a los firewall pfSense, conseguimos un cluster seguro, donde bloqueamos todo el tráfico menos el tráfico dentro de la red LAN. 
	\item \textbf{Objetivo 2}. Replicar infraestructura asegurando bajos tiempos de creación. Como hemos visto, el Objetivo 1 se ha conseguido utilizando un playbook de Ansible. Esto hace que inmediatamente se cumpla el objetivo 2, ya que este playbook nos va a permitir replicar la infraestructura en nuevos nodos. De nuevo en la sección \nameref{analisis_tiempos} podemos confirmar que hemos mejorado sustancialmente los tiempos de creación de infraestructura.
	\item \textbf{Objetivo 3}. Crear y desplegar servicios necesarios para una empresa de aplicaciones web asegurando bajos tiempos de creación. Todos los servicios necesarios para este proyecto, han sido desplegados y configurados con playbooks de Ansible. Esto permite crear y recrear los servicios en cualquier momento, asegurando un estado definido de cada servicio. El análisis de los tiempos se efectúa en la sección \nameref{analisis_tiempos}.
	\item \textbf{Objetivo 4}. Crear cauces rápidos y seguros para la creación de nuevo software. Con la ayuda de GitLab CI/CD y Docker, hemos conseguido crear un cauce de integración continua y despliegue continuo donde desplegar de forma segura los nuevos cambios en las aplicaciones. También se efectúa un análisis de la mejora en tiempos respecto a antes del proyecto en la sección \nameref{analisis_tiempos}.
\end{itemize}

\section{Análisis mejora tiempos}
\label{analisis_tiempos} 
\subsection{Tiempo estimado creación infraestructura antes del proyecto}

\subsection{Tiempo estimado creación infraestructura después del proyecto}
\subsection{Tiempo estimado creación servicios antes del proyecto}
\subsection{Tiempo estimado creación servicios después del proyecto}
\subsection{Tiempo estimado integración nuevos cambios en aplicación antes del proyecto}
\subsection{Tiempo estimado integración nuevos cambios en aplicación después del proyecto}


\section{Futuro del proyecto}
\begin{text}
	A continuación se hace un pequeño análisis de cómo puede evolucionar el proyecto en un futuro. De tener éxito y intentar aplicarlo a empresas, hay muchos aspectos del proyecto que se pueden mejorar o añadir. 
\end{text}
\subsection{Adaptación proyecto distintas plataformas}
\begin{text}
	Una de las principales desventajas que tiene este proyecto, es que está limitado su instalación y uso al sistema operativo Debian 10 buster. Se podría ampliar a otros entornos basados en debian pero actualmente, únicamente está comprobado y probado que funciona en Debian 10 buster. \\
	En caso de que el proyecto prospere, se debería ampliar la compatibilidad con distintas distribuciones de Linux como Ubuntu o CentOS que son de las más usadas. De esta forma se podría llegar a más público y hacer más atractivo el proyecto.
\end{text}
\subsection{Creación WUI para el despliegue de playbooks}
\begin{text}
	Dado el carácter técnico del proyecto, un usuario normal que utilizara este proyecto como servicio, tendría que aprender como funcionan los playbooks de Ansible en los que están basados este proyecto... Esto no es muy atractivo para posibles usuarios, ya que tendrían que tener conocimiento de cómo funcionan los playbooks, los roles... Para solventar el problema anterior, se sugiere la implementación de una interfaz de usuario, mediante la cual poder elegir que tareas ejecutar. Por ejemplo, se podrían crear formularios para insertar las variables necesarias y un selector donde elegir la tarea a desplegar. De esta forma el usuario no tendría que conocer como funcionan los playbooks o los roles de Ansible, simplemente mediante una interfaz intuitiva sabría que hace cada uno pero no como lo hacen. \\
	Esta parte del futuro del proyecto me parece especialmente interesante ya que una interfaz de usuario hace muy atractivo el producto. A continuación se muestra una posible interfaz
\end{text}


%\input{capitulos/07_Modelo de negocio}
%
\chapter {Modelo de negocio}

\section{Introducción}
	\begin{text}
		Introducimos un poco cómo se podría ganar dinero con este proyecto.
	\end{text}

\section{Inversión inicial}
	\begin{text}
		Explicamos la inversión inicial necesaria para hacer que este proyecto sea el centro de una empresa.
	\end{text}

\section{Hosting para empresas}
	\begin{text}
		Explicamos cómo podemos proveer infraestructura para empresas que se dediquen a desarrollo de software, alternativas a soluciones cloud.
	\end{text}

\section{Aplicación cauces desarrollo para empresas}
\begin{text}
	Explicamos cómo la empresa podría proporcionar servicios de adaptación de otras empresas a los nuevos modelos de desarrolo de software, haciendo más seguro y rápido la entrega del producto final.
\end{text}
%
%\input{capitulos/07_Pruebas}
%
%
%%\chapter{Conclusiones y Trabajos Futuros}
%
%
\nocite{*}
\bibliography{bibliografia/bibliografia}\addcontentsline{toc}{chapter}{Bibliografía}
\bibliographystyle{ieeetr}
%
%\appendix
%\input{apendices/manual_usuario/manual_usuario}
%%\input{apendices/paper/paper}
%\input{glosario/entradas_glosario}
% \addcontentsline{toc}{chapter}{Glosario}
% \printglossary
\chapter*{}
\thispagestyle{empty}

\end{document}
